%!TEX root=../document.tex

\section{Ergebnisse}
\label{sec:Ergebnisse}

\subsection{Vorgehensweise}
\subsubsection{MockUp}
Zuerst wurden die Mockup-Klassen erstellt. Eine MockUp Klasse dient dazu, eine bestimmte Funktionalität vorzugeben um somit die Komposition des Programmes zu testen. Beispielweise, wenn eine Klasse normalerweise ein File abspielen würde, macht eine MockUp Klasse nichts anderes als so zu tun als würde es das File abspielen indem es in der Konsole ausgibt \verb|File wird abgespielt|.

\subsubsection{MusikStueckFabrik}
Nachdem die simplen MockUp Klassen erstellt wurden, konnte man sich gleich an das tatsächliche Programm antasten. Der erste Schritt war, anstatt per Hand Lieder zu erstellen, ein gewünschtes Directory durchzugehen und dort alle Musik-Lieder heraus zu suchen. 

Umgesetzt wurde dies, klassisch wie man es meistens in Python macht, mit \verb|os.walk()| und einer Rekursion. Bei jedem File wird anschließlich nachgeschaut, ob die extension einem Musik-File gleicht, dann werden Meta-Daten ausgelesen und schließlich das MusikStueck erstellt. 

Falls keine passenden Lieder gefunden wurden, wird eine Fehlernachricht ausgegeben

\begin{lstlisting}[language=Python]
def lade_musik(self):
    """
    Fuegt der Playlistt MockupMusikstuecke hinzu welche automatisch ausgelesen werden
    
    :return: None
    """ 
    # iterate through given directory
    for dirname, subdirs, files in os.walk(self.dir):
	    for filename in files:
		    # get the file extension of each file
		    extension = os.path.splitext(filename)[1][1:]
		    # check if the extension is a music file
		    if extension.lower() in ["mp3","wma", "wav","ra", "ram", "rm", "mid", "flac", "ogg"]:
				# get each mp3 file in this directory
			    file = pyglet.media.load(dirname + os.path.sep + filename)
    
			    # get each specific info needed
			    songlaenge = file.duration
			    songtitel = file.info.title.decode()
			    songinterpret = file.info.author.decode()
			    songalbum = file.info.album.decode()
			    # append each song with the informations to the playlist and the gui object
			    self.playlist.append(MusikstueckFile(file, songlaenge, songtitel, songinterpret, songalbum, self.updatefunction))
		    else:
			    print("%s is not a music file!" % filename)
    if len(self.playlist) == 0:
	    # if no music file was found, print out error message and exit program
	    print("No suitable Files found. Please restart Program and choose another Folder")
	    sys.exit(0)
\end{lstlisting}

\clearpage

\subsubsection{MusikStueck}
Der nächste Schritt war es das MusikStueckFile zu verändern, also jene Klasse die tatsächlich das File abspielt. Diese Klasse erbt von MusikStueck, und eigentlich muss nur die abstrakte Methode \verb|abspielen()| überschrieben werden, aber weil noch zusätzliche Informationen benötigt werden, wurde dem Konstruktur noch 3 Parameter hinzugefügt:

\begin{lstlisting}[language=Python]
def __init__(self, file, laenge, titel, interpret, album, update_function):
    """
    Adds 2 additional parameters to the super constructor
    
    :param file: Thile file object which gets played
    
    :param laenge: The length of the mp3 file in order to pause the playlist for this amount
    
    :param titel: Title of the song, read out of the mp3 file
    
    :param interpret: Artist of the song, read out of the mp3 file
    
    :param album: Album of the song, read out of the mp3 file
    
    :param updatefunction: function of the gui object to update the data
    """
    # call super constructor
    Musikstueck.__init__(self, titel, interpret, album)
    # set the 3 additional parameters to class attributes
    self.file = file
    self.laenge = laenge
    self.update_function = update_function
\end{lstlisting}

In der \verb|abspielen()| methode wird lediglich das pyglet.media File abgespielt, die Methode aufgerufen welche im Gui die Daten ändert und gewartet bis das Lied zu Ende ist.

\begin{lstlisting}[language=Python]
def abspielen(self):
    """
    Play the file object given and update information
    
    In order for the playlist not to play every song at the same time, everytime a song is played, time.sleep for the length of the song
    
    :return: None
    """
    self.file.play()
    self.update_function(str(self.interpret), str(self.titel), str(self.album))
    time.sleep(self.laenge)
\end{lstlisting}


\subsection{Probleme}
\subsubsection{avbin.dll}
pyglet verwendet um MusikFiles abzuspielen, abgesehen von .wma, eine externe \verb|dynamic link library| die \verb|avbin.dll| heißt. Diese kann man im Internet überall finden, es war jedoch eine Herausforderung herauzufinden welche .dll man herunterladen muss und wo sie hingehört oder wie man sie in das System einfügt. 

Wie es für mich funktioniert hat, war indem ich eine 32bit Version heruntergeladen habe, namens \verb|avbin.dll| (ohne 64) und diese in \verb|C:\Windows\System| verschoben habe. Somit konnte pyglet die Library verwenden.

\clearpage

\subsubsection{Gui Freezing}
Wie erstmals das Gui hinzugefügt wurde, hat das Gui nicht reagiert und hat immer zu einem Absturz des Programmes geführt. Problem war folgendes: 

In meiner \verb|MusikStueckFile| Klasse, wo das tatsächliche File abgespielt wird, ist ein \verb|time.sleep()| eingebaut, damit während einem Song gewartet wird damit nicht alle Songs in der Playlist gleichzeitig abgespielt werden. Dies hat dazu geführt dass der \verb|Main Thread| immer in einem Wartezustand ist, und somit das Gui auch wartet und somit nicht reagiert.

Lösung zu diesem Problem war, in der Main Klasse eine Thread-Klasse zu erzeugen, welche die Factory initialisiert und startet. Dies führt dazu dass die GUI und der Factory-Prozess, in welchem gewartet wird, in verschieden Threads realisiert ist und so unabhängig voneinander agieren können. 

\subsubsection{Programm Beenden}
Wenn die GUI geschlossen wurde, ist im Hintergrund trotzdem das Programm weitergelaufen. Dies war ein Problem infolge der Lösung des anderen Problems, weil ja diese Prozesse nun in verschiedenen Threads ablaufen.

Dieses Problem konnte mithilfe von \verb|Pierre Rieger| leicht behoben werden, indem im \verb|FactoryThread| das Klassen-Attribut \verb|self.daemon| auf True gesetzt wurde. Erklärung wie es im Source-Code steht:

\begin{lstlisting}[language=Python]
class FactoryThread(Thread):
	# factorythread for starting the factory process aka playing the music
	# has to implemented as thread and can't be written plain in the main thread because
	# else the gui freezes
	def __init__(self,dir,update_function):
		"""
		Calls super constructor and initializes class attributes
		
		if self.deamon is set to True, the thread automatically becomes a 'Deamon-Thread'
		this basically means that if no other Threads are left, this Deamon Threads also automatically
		closes itself, which is incredibly helpful because the gui Thread gets terminated by closing the window
		but the factory_thread will always keep on running unless its a daemon thread
		
		:param dir: directory with which the factory gets initialized
		
		:param update_function:
		"""
		Thread.__init__(self)
		self.dir = dir
		self.update_function = update_function
		self.daemon = True
\end{lstlisting}