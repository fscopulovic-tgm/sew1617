%!TEX root=../document.tex

\section{Einführung}
Schreibe ein Programm, welches einen Music Player einerseits simuliert und andererseits implementiert!

\subsection{Grundanforderungen}
\begin{itemize}
\item Verwende die vorgegebenen Klassen:
	\begin{itemize}
		\item main: Erzeugt entweder eine Mockup-Fabrik oder eine echte Fabrik
		\item Musikstueck: Abstrakte Produkt-Klasse
		\item MusikdatenbankFabrik: Abstrakte Fabrik mit Fabrik-Methode \verb|lade_musik|
	\end{itemize}
\item Eine eigene Mockup-Klasse mockt die abspielen-Methode mit einer simplen Ausgabe (Mockup-Produkt), z.B. Sie hören den Titel XXX von YYY aus dem Album ZZZ
\item Eine eigene Mockup-Klasse erzeugt einige beliebige Mockup-Produkte (Mockup-Fabrik)
\item Eine eigene Klasse spielt die Musik ab (File-Produkt)
\item Eine eigene Klasse erzeugt die echten Produkte, indem sie nach mp3-Files sucht und für jedes mp3-File ein Produkt erzeugt (File-Fabrik)
\item Kommentare und Sphinx-Dokumentation
\item Protokoll
\end{itemize}
\textbf{Tipp}: Verwende zum Abspielen die Library pyglet. Achtung: pyglet.app.run() endet nicht automatisch, sondern muss über ein Callback nach dem Song über pyglet.app.exit() wieder beendet werden!

\subsection{Erweiterungen}
\begin{itemize}
\item Recherchiere, wie aus Musikdateien automatisch Titel, Album und Interpret ausgelesen werden können
\item Füge die Funktionalität deiner echten Produktfamilie hinzu, sodass die Informationen ebenfalls angezeigt werden können
\item Baue eine (sehr simple) GUI für deinen Music Player
\end{itemize}
\clearpage
